%%%%%%%%%%%%%%%%%%%%%%%%%%%%% Define Article %%%%%%%%%%%%%%%%%%%%%%%%%%%%%%%%%%
\documentclass[addpoints]{exam}
%%%%%%%%%%%%%%%%%%%%%%%%%%%%%%%%%%%%%%%%%%%%%%%%%%%%%%%%%%%%%%%%%%%%%%%%%%%%%%%

%%%%%%%%%%%%%%%%%%%%%%%%%%%%% Using Packages %%%%%%%%%%%%%%%%%%%%%%%%%%%%%%%%%%
\usepackage{geometry}
\usepackage{graphicx}
\usepackage{amssymb}
\usepackage{amsmath}
\usepackage{amsthm}
\usepackage{empheq}
\usepackage{mdframed}
\usepackage{booktabs}
\usepackage{lipsum}
\usepackage{graphicx}
\usepackage{color}
\usepackage{psfrag}
\usepackage{pgfplots}
\usepackage{bm}
%%%%%%%%%%%%%%%%%%%%%%%%%%%%%%%%%%%%%%%%%%%%%%%%%%%%%%%%%%%%%%%%%%%%%%%%%%%%%%%

% Other Settings

%%%%%%%%%%%%%%%%%%%%%%%%%% Page Setting %%%%%%%%%%%%%%%%%%%%%%%%%%%%%%%%%%%%%%%
\geometry{a4paper}

%%%%%%%%%%%%%%%%%%%%%%%%%% Define some useful colors %%%%%%%%%%%%%%%%%%%%%%%%%%
\definecolor{ocre}{RGB}{243,102,25}
\definecolor{mygray}{RGB}{243,243,244}
\definecolor{deepGreen}{RGB}{26,111,0}
\definecolor{shallowGreen}{RGB}{235,255,255}
\definecolor{deepBlue}{RGB}{61,124,222}
\definecolor{shallowBlue}{RGB}{235,249,255}
%%%%%%%%%%%%%%%%%%%%%%%%%%%%%%%%%%%%%%%%%%%%%%%%%%%%%%%%%%%%%%%%%%%%%%%%%%%%%%%

%%%%%%%%%%%%%%%%%%%%%%%%%% Define an orangebox command %%%%%%%%%%%%%%%%%%%%%%%%
\newcommand\orangebox[1]{\fcolorbox{ocre}{mygray}{\hspace{1em}#1\hspace{1em}}}
%%%%%%%%%%%%%%%%%%%%%%%%%%%%%%%%%%%%%%%%%%%%%%%%%%%%%%%%%%%%%%%%%%%%%%%%%%%%%%%

%%%%%%%%%%%%%%%%%%%%%%%%%%%% English Environments %%%%%%%%%%%%%%%%%%%%%%%%%%%%%
\newtheoremstyle{mytheoremstyle}{3pt}{3pt}{\normalfont}{0cm}{\rmfamily\bfseries}{}{1em}{{\color{black}\thmname{#1}~\thmnumber{#2}}\thmnote{\,--\,#3}}
\newtheoremstyle{myproblemstyle}{3pt}{3pt}{\normalfont}{0cm}{\rmfamily\bfseries}{}{1em}{{\color{black}\thmname{#1}~\thmnumber{#2}}\thmnote{\,--\,#3}}
\theoremstyle{mytheoremstyle}
\newmdtheoremenv[linewidth=1pt,backgroundcolor=shallowGreen,linecolor=deepGreen,leftmargin=0pt,innerleftmargin=20pt,innerrightmargin=20pt,]{theorem}{Theorem}
\theoremstyle{mytheoremstyle}
\newmdtheoremenv[linewidth=1pt,backgroundcolor=shallowBlue,linecolor=deepBlue,leftmargin=0pt,innerleftmargin=20pt,innerrightmargin=20pt,]{definition}{Definition}
\theoremstyle{myproblemstyle}
\newmdtheoremenv[linecolor=black,leftmargin=0pt,innerleftmargin=10pt,innerrightmargin=10pt,]{problem}{Problem}[section]
%%%%%%%%%%%%%%%%%%%%%%%%%%%%%%%%%%%%%%%%%%%%%%%%%%%%%%%%%%%%%%%%%%%%%%%%%%%%%%%

%%%%%%%%%%%%%%%%%%%%%%%%%%%%%%% Plotting Settings %%%%%%%%%%%%%%%%%%%%%%%%%%%%%
\usepgfplotslibrary{colorbrewer}
\pgfplotsset{width=8cm,compat=1.9}
%%%%%%%%%%%%%%%%%%%%%%%%%%%%%%%%%%%%%%%%%%%%%%%%%%%%%%%%%%%%%%%%%%%%%%%%%%%%%%%

%%%%%%%%%%%%%%%%%%%%%%%%%%%%%%% Title & Author %%%%%%%%%%%%%%%%%%%%%%%%%%%%%%%%
\title{Week 9 Problems}
\author{Muhammad Meesum Ali Qazalbash}
%%%%%%%%%%%%%%%%%%%%%%%%%%%%%%%%%%%%%%%%%%%%%%%%%%%%%%%%%%%%%%%%%%%%%%%%%%%%%%%
\printanswers

\begin{document}
	\maketitle
	\begin{center}
		\gradetable[h][questions]
	\end{center}

	% \begin{definition}[\(\varepsilon\)-Neighborhood]
	% 	If \(\mathbb{F}\) is an ordered field, \(c\in\mathbb{F}\) then \(\varepsilon\)-neighborhood of \(c\); \(\aleph_{\varepsilon}(c)\) is defined as
	% 	\[\aleph_{\varepsilon}(c):=\{x:x\in\mathbb{F}\land |x-c|<\varepsilon\}\]
	% \end{definition}

	% \begin{definition}[Neighborhood]
	% 	A neighborhood of \(c\in(\mathbb{F},\mathbb{P})\) is any set \(S\subseteq\mathbb{F}\ni S\) contains some \(\varepsilon\)-neighborhood of \(c\) i.e. \(\exists \varepsilon>0\ni\aleph_{\varepsilon}(c)\subseteq S\).
	% \end{definition}

\begin{questions}
	\question[1] Suppose \(U\) is a neighborhood of a point \(x\) and that \(U\subseteq V\). Show that \(V\) is a neighborhood of \(x\).
	\begin{solution}
		\(U\) is a neighborhood of \(x\) that means there exists \(\varepsilon>0\) such that \(\aleph_{\varepsilon}(x)\subseteq U\). Now, \(U\subseteq V\) means \(\aleph_{\varepsilon}(x)\subseteq V\). Therefore, \(V\) is a neighborhood of \(x\). \hfill\(\blacksquare\)
	\end{solution}

	\question[1] \begin{parts}
		\part Suppose \(U\) and \(V\) are neighborhoods of a point \(x\). Show that \(U\cap V\) is a neighborhood of \(x\).
		\begin{solution}
			\(U\) and \(V\) are neighborhoods of \(x\) that means there exists \(\varepsilon_{1}>0\) and \(\varepsilon_{2}>0\) such that \(\aleph_{\varepsilon_{1}}(x)\subseteq U\) and \(\aleph_{\varepsilon_{2}}(x)\subseteq V\). Now, \(x\in U\cap V\) we can define a \(\varepsilon'=\min\{\varepsilon_1,\varepsilon_2\}\). \(\aleph_{\varepsilon'}(x)\subseteq U\cap V\) is an \(\varepsilon\)-neighborhood of \(x\). Therefore, \(U\cap V\) is a neighborhood of \(x\). \hfill\(\blacksquare\)
		\end{solution}

		\part Show that (a) remains true for a finite collection of neighborhoods.
		\begin{solution}
			Let \(U_{1},U_{2},\dots,U_{n}\) be neighborhoods of \(x\) then \(\forall i\in\{1,2,\cdots,n\},\exists\varepsilon_{i}>0\ni\aleph_{\varepsilon_{i}}(x)\subseteq U_{i}\). Now, \(\displaystyle x\in\bigcap_{i=1}^{n}U_{i}\), we can define a \(\varepsilon'=\min\{\varepsilon_1,\varepsilon_2,\dots,\varepsilon_n\}\). \(\displaystyle\aleph_{\varepsilon'}(x)\subseteq\bigcap_{i=1}^{n}U_{i}\) is an \(\varepsilon\)-neighborhood of \(x\). Therefore, \(\displaystyle\bigcap_{i=1}^{n}U_{i}\) is a neighborhood of \(x\). \hfill\(\blacksquare\)
		\end{solution}

		\part Does (a) remains true for an infinite collection of neighborhoods?
		\begin{solution}
			
		\end{solution}
	\end{parts}

	\question[1] \begin{parts}
		\part If \(S\) and \(T\) are bounded sets, show that \(S\cap T\) are bounded.
		\begin{solution}

		\end{solution}

		\part If \(S\) and \(T\) are as in (a), show that \(\sup \left(S\cup T\right)=\max \{\sup S,\sup T\}\) (be sure to justify use of \(\max\)).
		\begin{solution}
		
		\end{solution}

		\part Is it true that \(\sup\left(S\cap T\right)=\min\{\sup S, \sup T\}\)?
		\begin{solution}
		
		\end{solution}

		\part Give a condition under which the equality in (c) would be true.
		\begin{solution}
		
		\end{solution}

		\part Let \(\{S_\alpha:\alpha\in\mathcal{A}\}\) be a collection of bounded sets (where \(\mathcal{A}\) is finite). Show that \(\displaystyle\bigcup_{\alpha\in\mathcal{A}}S_\alpha\) is bounded.
		\begin{solution}
		
		\end{solution}

		\part Let \(\{S_\alpha:\alpha\in\mathcal{A}\}\) be a collection of bounded sets (where \(\mathcal{A}\) is infinite). Is \(\displaystyle\bigcup_{\alpha\in\mathcal{A}}S_\alpha\) necessarily bounded?
		\begin{solution}
		
		\end{solution}
	\end{parts}

	\question[1] \begin{parts}
		\part Suppose \(x_k\) is a real number for \(k=1,2,\cdots\), and there is a positive number \(\varepsilon\) so that \(x_k>\varepsilon\) for each \(k\). If \(B\) is any real number, show that there is a natural number \(n\) so that \(x_1+x_2+\cdots+x_n>B\).
		\begin{solution}
		
		\end{solution}

		\part Show that this need not be the case if we assume only that \(x_n>0\) for all \(n\).
		\begin{solution}
		
		\end{solution}
	\end{parts}
\end{questions}
	
\end{document}