%%%%%%%%%%%%%%%%%%%%%%%%%%%%% Define Article %%%%%%%%%%%%%%%%%%%%%%%%%%%%%%%%%%
\documentclass[addpoints]{exam}
%%%%%%%%%%%%%%%%%%%%%%%%%%%%%%%%%%%%%%%%%%%%%%%%%%%%%%%%%%%%%%%%%%%%%%%%%%%%%%%

%%%%%%%%%%%%%%%%%%%%%%%%%%%%% Using Packages %%%%%%%%%%%%%%%%%%%%%%%%%%%%%%%%%%
\usepackage{geometry}
\usepackage{graphicx}
\usepackage{amssymb}
\usepackage{amsmath}
\usepackage{amsthm}
\usepackage{empheq}
\usepackage{mdframed}
\usepackage{booktabs}
\usepackage{lipsum}
\usepackage{graphicx}
\usepackage{color}
\usepackage{psfrag}
\usepackage{pgfplots}
\usepackage{bm}

\printanswers
\renewcommand{\solutiontitle}{\textit{Proof. }}
\title{Week 13 Problems}
\author{Muhammad Meesum Ali Qazalbash}
%%%%%%%%%%%%%%%%%%%%%%%%%%%%%%%%%%%%%%%%%%%%%%%%%%%%%%%%%%%%%%%%%%%%%%%%%%%%%%%

\begin{document}
\maketitle
\center\gradetable[h]
\begin{questions}
	\question[1] If \(S\) is a set that is bounded above, show that \(\sup{S}\) is either an element of \(S\) or is a cluster point of \(S\).
	\begin{solution}
		If \(\sup{S}\in S\), then you are done. Suppose \(\sup{S}\notin S\). Let, \(x=\sup{S}\). We argue now that \(x\) is a cluster point. By definition, it is enough to argue that for any \(\varepsilon>0\), there is a point \(z\neq x\) such that \(z\in S\) and \(z\) lies in the \(\varepsilon\)-neighborhood of \(x\). To this end, note that \(x-\varepsilon<x\) and because \(x\) is the least upper bound of \(S\), there is \(z\in S\) such that \(z>x-\varepsilon\). Because \(z\in S\), you have \(z\leq x\) so that \(z\in(x-\varepsilon,x]\subseteq (x-\varepsilon,x+\varepsilon)\). Finally, because \(z\in S\) and \(x\notin S\), \(z\neq x\). \hfill\qed
	\end{solution}

	\question[4] We may say that \(x\) is a "right" cluster point of a set \(S\) if, for any \(\varepsilon>0,S\cap(x,x+\varepsilon)\neq\emptyset\), and similarly for "left" cluster points. We could also insist that these intersections be infinite.
	\begin{parts}
		\part Show that the two definitions of right cluster point suggested above are equivalent, and the two definition of left cluster point suggested above are equivalent.
		\begin{solution}
			We have to prove that,
			\[
				\forall\varepsilon>0, S\cap(x,x+\varepsilon)\neq\emptyset\iff S\cap(x,x+\varepsilon)\mbox{ is infinite}
			\]
			We would first consider the trivial case which is,
			\[
				\forall\varepsilon>0,S\cap(x,x+\varepsilon)\mbox{ is infinite}\implies S\cap(x,x+\varepsilon)\neq\emptyset
			\]
			If the intersection is infinite then it is not empty. Now we would prove the other direction,
			\[
				\forall\varepsilon>0,S\cap(x,x+\varepsilon)\neq\emptyset\implies S\cap(x,x+\varepsilon)\mbox{ is infinite}
			\]
			Let \(x_0\in S\cap(x,x+\varepsilon)\) and \(\forall n\in\mathbb{N}\),
			\[
				\varepsilon_{n}:=\mid x_{n-1}-x\mid >0
			\]
			\[
				x_n\in S\cap(x,x+\varepsilon_n)
			\]
			Each \(\varepsilon\) is smaller than the previous one,
			\[
				\varepsilon_n\leq\varepsilon_{n-1}
			\]
			\[
				x_n\in S\cap(x,x+\varepsilon_n)\subseteq S\cap(x,x+\varepsilon_{n-1})\subseteq S\cap(x,x+\varepsilon)
			\]
			\[
				\therefore x_n\in S\cap(x,x+\varepsilon)
			\]
			\[
				\implies S\cap(x,x+\varepsilon)\mbox{ is infinite}
			\]
			Similarly, it is for left cluster points.\hfill\qed
		\end{solution}
		\part Show that every right cluster point is a cluster point, but that not every cluster point is a right cluster point. Similarly for left cluster points.
		\begin{solution}
			We have prove this in (a) that every right cluster point is a cluster point. Now we would prove that not every cluster point is a right cluster point. Consider a set
			\[S:=\left\lbrace-\frac{1}{n}\bigg| n\in\mathbb{N}\right\rbrace\]
			0 is the cluster point of \(S\) but it is not a right cluster point. Similarly, consider another set,
			\[S:=\left\lbrace\frac{1}{n}\bigg| n\in\mathbb{N}\right\rbrace\]
			0 is the cluster point of \(S\) but it is not a left cluster point.\hfill\qed
		\end{solution}

		\part Examine the examples of cluster points in the chapter. Which are right cluster points and which are left cluster points?
		\begin{solution}
			\(\displaystyle\frac{1}{2}\) is both a right and left cluster point of \((0, 1)\). 0 is a right cluster point of \(\displaystyle H = \left\lbrace 1, \frac{1}{2}, \frac{1}{3}, \cdots \right\rbrace\).
		\end{solution}
		\newpage
		\part If \(x\) is a cluster point of \(S\), must it be the case that \(x\) is either a right cluster point or a left cluster point of \(S\)?
		\begin{solution}
			We have to prove that,
			\[
				x\mbox{ is a CP of }S\implies x\mbox{ is either a right CP or a left CP of }S
			\]
			It is equivalent of proving,
			\[
				x\mbox{ is a CP of }S \mbox{ and } x \mbox{ not left CP of }S\implies x\mbox{ is either a right CP}
			\]
			\[
				\forall\varepsilon>0,S\cap(x,x+\varepsilon)\mbox{ is infinite}\implies x\mbox{ is a right CP of }S
			\]\hfill\qed
		\end{solution}
	\end{parts}
	\question[2] \begin{parts}
		\part Show that \(\lim{x_n}=L\) if and only if the sequence described by \(x_1, L, x_2, L, x_3, L, \cdots\), converges.
		\begin{solution}
			We have to prove that,
			\[
				\lim{x_n}=L\iff x_1, L, x_2, L, x_3, L, \cdots \mbox{ converges}
			\]
			First we will prove that,
			\[
				\lim{x_n}=L\implies x_1, L, x_2, L, x_3, L, \cdots \mbox{ converges}
			\]
			We know that,
			\[
				\lim{x_n}=L\implies\forall\varepsilon>0,\exists N\in\mathbb{N},\forall n>N, x_n\in\aleph_\varepsilon(L)
			\]
			We also know that \(L\in\aleph_\varepsilon(L)\),
			\[
				\forall\varepsilon>0,\exists N\in\mathbb{N},\forall n>N, x_n\in\aleph_\varepsilon(L)\land L\in\aleph_\varepsilon(L)
			\]
			Therefore, \(
			x_1, L, x_2, L, x_3, L, \cdots
			\) converges.
			Now we will prove that,
			\[
				x_1, L, x_2, L, x_3, L, \cdots \mbox{ converges}\implies\lim{x_n}=L
			\]
			For any convergent sequence, its subsequence is also converges and converges to the same limit. \(L,L,L,\cdots\) converges to \(L\) and it is a subsequence of \(x_1,L,x_2,L,x_3,L,\cdots\), therefore \(x_1,x_2,x_3,\cdots\) must also converge to \(L\).
			\hfill\qed
		\end{solution}
		\newpage
		\part Show that \(\lim{x_n}=\lim{y_n}\) if and only if the sequence described by \(x_1, y_1, x_2, y_2, \cdots\), converges.
		\begin{solution}
			We have to prove that,
			\[
				\lim{x_n}=\lim{y_n}\iff x_1, y_1, x_2, y_2, \cdots \mbox{ converges}
			\]
			First we will prove that,
			\[
				\lim{x_n}=\lim{y_n}\implies x_1, y_1, x_2, y_2, \cdots \mbox{ converges}
			\]
			We will prove this by contradiction. Suppose that, \(\lim{x_n}\) and \(y_n\) exists and they are equal but the sequence \(x_1, y_1, x_2, y_2, \cdots\) does not converge. If two subsequence of the a sequence converges to different points then the sequence is divergent. \(x_1,x_2,x_3,\cdots\) and \(y_1,y_2,y_3,\cdots\) is a subsequence of \(x_1,y_1,x_2,y_2,x_3,y_3,\cdots\) and the subsequences converges to the same point this means the sequence is convergent. But we have assumed that the sequence is divergent. Therefore, the assumption is wrong. Therefore, the sequence \(x_1, y_1, x_2, y_2, \cdots\) converges.
			Now we will prove that,
			\[
				x_1, y_1, x_2, y_2, \cdots \mbox{ converges}\implies\lim{x_n}=\lim{y_n}
			\]
			This is trivial. If the sequence converges then its limit exists and the limit of the subsequence is also the same limit. \((x_n)\) and \((y_n)\) is a subsequence of \(x_1, y_1, x_2, y_2, \cdots\). Therefore, the limit of the sequence is the same as the limit of the subsequence.
			\hfill\qed
		\end{solution}
	\end{parts}
	\question[4] \begin{parts}
		\part Show that if \(\lim{x_n}=L\), then \(\lim{\mid x_n\mid}=\mid L\mid\).
		\begin{solution}
			\begin{align*}
				\lim{x_n}=L & \implies\;\forall\varepsilon>0,\exists N\in\mathbb{N},\forall n>N,x_n\in\aleph_\varepsilon(L)                                   \\
				            & \implies\;\forall\varepsilon>0,\exists N\in\mathbb{N},\forall n>N,\mid x_n-L\mid<\varepsilon                                    \\
				            & \implies\;\forall\varepsilon>0,\exists N\in\mathbb{N},\forall n>N,\mid\mid x_n\mid-\mid L\mid\mid\leq\mid x_n-L\mid<\varepsilon \\
				            & \implies\;\forall\varepsilon>0,\exists N\in\mathbb{N},\forall n>N,\mid\mid x_n\mid-\mid L\mid\mid<\varepsilon                   \\
				            & \implies\;\lim{\mid x_n\mid}=\mid L\mid
			\end{align*}
			\hfill\qed
		\end{solution}
		\part Show that the converse of (a) is not true.
		\begin{solution}
			Let \(\displaystyle\mid x_n\mid=1+\frac{1}{2^n}\) then, \(\lim{\mid x_n\mid}=1=\mid L\mid\). But \(x_n\) can be equal to \(\displaystyle 1+\frac{1}{2^n}\) or \(\displaystyle-1-\frac{1}{2^n}\) and the limit of the respective sequence are different. Therefore, the converse of (a) is not true.\hfill\qed
		\end{solution}
		\newpage
		\part Show that if \(\lim{\mid x_n\mid}=0\), then \(\lim{x_n}=0\).
		\begin{solution}
			\begin{align*}
				\lim{\mid x_n\mid}=0 & \implies\;\forall\varepsilon>0,\exists N\in\mathbb{N},\forall n>N,\mid\mid x_n\mid\mid<\varepsilon \\
				                     & \implies\;\forall\varepsilon>0,\exists N\in\mathbb{N},\forall n>N,\mid x_n\mid<\varepsilon         \\
				                     & \implies\;\lim{x_n}=0
			\end{align*}
			\hfill\qed
		\end{solution}
		\part What property of 0 makes (c) true?
		\begin{solution}
			For \(\mid a\mid=0\), \(a\) is unique, that is \(a=0\).\hfill\qed
		\end{solution}
	\end{parts}
	\question[4] \begin{parts}
		\part Give an example to show that it is possible to have \(\lim(x_n + y_n)\) exist without having \(\lim x_n\) or \(\lim y_n\) exist.
		\part Give an example to show that it is possible to have \(\lim(x_ny_n)\) exist without having \(\lim x_n\) or \(\lim y_n\) exist.
		\begin{solution}
			Consider two sequences \(x_{n}:=\{(-1)^n\}_{n\in\mathbb{N}}\) and \(y_{n}:=\{(-1)^{n+1}\}_{n\in\mathbb{N}}\) then, \(\nexists\lim{x_n}\) and \(\nexists\lim{y_n}\). But, \(x_n+y_n=\{0\}_{n\in\mathbb{N}}\) and \(\exists\lim{(x_n+y_n)}=0\). And, \(x_ny_n=\{-1\}_{n\in\mathbb{N}}\) and \(\exists\lim{(x_ny_n)}=-1\).
			\hfill\qed
		\end{solution}
		\part If \(\lim(x_n + y_n)\) exists and \(\lim x_n\) exists, must it be the case that \(\lim y_n\) exists?
		\begin{solution}
			\[
				\forall x_n,y_n,\exists \lim{(x_n+y_n)} \land \exists \lim{x_n}\implies \exists \lim{y_n}
			\]
			Let \(\lim{(x_n+y_n)}=L\) and \(\lim{x_n}=p\)
			\[
				\exists\lim{x_n}=p\implies\forall\varepsilon_p>0,\exists N\in\mathbb{N}\ni\forall n>N,\mid x_n-p\mid<\varepsilon_p
			\]
			\[
				\exists\lim{(x_n+y_n)}=L\implies\forall\varepsilon_L>0,\exists N\in\mathbb{N}\ni\forall n>N,\mid x_n+y_n-L\mid<\varepsilon_L
			\]
			Now we will manipulate (algebrically) the above two inequalities to get \(\exists\lim{y_n}\).
			\begin{align*}
				p-\varepsilon_p                                                        & <x_n<p+\varepsilon_p                                             \\
				\implies-\varepsilon_p-p                                               & <-x_n<\varepsilon_p-p                                            \\
				L-\varepsilon_L                                                        & <x_n+y_n<L+\varepsilon_L                                         \\
				\implies L-p-(\underbrace{\varepsilon_L+\varepsilon_p}_{\varepsilon'}) & <y_n<L-p+\underbrace{\varepsilon_L+\varepsilon_p}_{\varepsilon'} \\
				\implies L-p-\varepsilon'                                              & <y_n<L-p+\varepsilon'                                            \\
				\implies \mid y_n - L + p \mid                                         & <\varepsilon'                                                    \\
				\implies \lim{y_n}                                                     & = L-p
			\end{align*}
			The \(\lim{y_n}\) must exists.
			\hfill\qed
		\end{solution}
		\part If \(\lim(x_ny_n)\) exists and \(\lim x_n\) exists, must it be the case that \(\lim y_n\) exists?
		\begin{solution}
			Consider two sequences \(\displaystyle x_n:=\left\lbrace\frac{1}{n^2}\right\rbrace_{n\in\mathbb{N}}\) and \(\displaystyle y_n:=\left\lbrace n\right\rbrace_{n\in\mathbb{N}}\). The product of these sequences is \(\displaystyle x_ny_n=\left\lbrace\frac{1}{n}\right\rbrace_{n\in\mathbb{N}}\). We have \(\lim{(x_ny_n)}=0\) and \(\lim{x_n}=0\). But, \(\nexists\lim{y_n}\).
			\hfill\qed
		\end{solution}
	\end{parts}
\end{questions}

\end{document}

% If \(\lim(x_n + y_n)\) and \(\lim x_n\) exists, Let \(\lim(x_n + y_n) = L\) and \(\lim x_n = m\).

% By the properties of limits, we know that \(\lim(x_n + y_n) = \text{lim}x_n + \text{lim}y_n \)
% \( \implies L = m + \text{lim}y_n \implies \text{lim}y_n = L - m \)

% Let \( L - m = n \) Then \(\lim y_n = n\)

% Since both \(L\) and  \(m\) exists, then \(n\) must also exist. Therefore, \(\lim y_n\) must exist.

% It can be better (complexer better) understood by considering,
% \[ |x_n - x| < \frac{\varepsilon}{2}, \forall n > m \]
% and

% \[ |y_n - y| < \frac{\varepsilon}{2}, \forall n > n \]

% where \(m\) and \(n\) are limits of \(x_n\) and \(y_n\).

% Consequently,
% \begin{align*}
% 	| (x_n + y_n) - (x + y) | = | (x_n - x) + (y_n - y) | \leq & \; |x_n - x| + |y_n - y|\\
% 	\leq & \; \frac{\varepsilon}{2} + \frac{\varepsilon}{2} = \varepsilon, \forall n \geq N = \max\{m, n\}
% \end{align*}

% Hence the sequence \(x_n + y_n\) converges and \(\lim(x_n + y_n) = \lim x_n + \lim y_n\) and they all exist.
% \hfill\qed